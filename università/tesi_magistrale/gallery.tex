
\section{Qualitative Analysis Gallery}
\label{sec:gallery}

This section provides an in-depth qualitative analysis of the model performances by examining eight representative samples from the test set. The selected examples cover a diverse range of visual complexity, spanning from iconic representations and simple geometric patterns to more abstract designs. This variety allows us to observe specific failure modes in baseline models---such as the hallucination of non-existent details or the misinterpretation of spatial relationships---while highlighting the strengths of our proposed SPE-based architectures. For each sample, we present the rendered SVG image alongside the ground truth caption and the varying descriptions generated by Florence-2, Qwen2-7B, Llama-3-8B, and our fine-tuned SPE variants (SPE+Qwen2 and SPE+Gemma). This side-by-side comparison offers tangible insight into how different architectures perceive and textualize vector graphics.

\begin{figure}[!ht]
    \centering
    \includegraphics[width=0.25\textwidth]{images/gallery/sample_4.png}
    \caption*{Sample 4: Circle with Square}
    \vspace{-1.0em}
    \scriptsize
    \setlength{\itemsep}{0pt}
    \setlength{\parskip}{0pt}
    \begin{description}
        \item[\textbf{Ground Truth}:] The image depicts a simple geometric shape, specifically a circle with a small square at the top. The circle is composed of a series of evenly spaced dots, which are uniformly distributed around its circumference. The square at the top of the circle is slightly smaller than the circle itself and is positioned directly above the center of the circle. The square is also made up of dots, similar to the circle, but it is oriented with its sides parallel to the horizontal and vertical axes of the image.
        \item[\textbf{BLIP-2}:] a black and white icon of a clock
        \item[\textbf{Florence-2}:] The image is a black and white icon of a ring. The ring is circular in shape and has multiple small black dots arranged in a circular pattern. At the top of the ring, there is a small black rectangle with a curved top. The background is white. The overall design is simple and minimalistic.
        \item[\textbf{Idefics3}:] The image contains a series of dots arranged in a circular pattern. The dots are evenly spaced and form a continuous loop. The loop is symmetrical, with the dots forming a perfect circle. The dots are black, and the background is white. The image does not contain any text or other objects. The arrangement of the dots suggests a pattern that could be interpreted as a symbol or a mathematical representation.
        \item[\textbf{BLIP-1-CPU}:] 
        \item[\textbf{Qwen2-7B}:] The SVG code describes intricate geometric shapes with precise dimensions. The primary focus is on a complex star-like figure made up of multiple overlapping circles and arcs. Here's the breakdown: \textbf{Star Shape}: At its core, there's a star pattern composed of 10-pointed stars formed by intersecting circles and arcs. Each point of the star is defined by a series of arcs that connect to form a symmetrical shape. \textbf{Circular Elements}: - There are four large circles arranged around the star pattern. - These circles have diameters ranging from 10 units to 15 units, creating a layered effect when combined with smaller concentric circles within them. \textbf{Smaller Circles/Arcs}: Numerous small circles and arcs are positioned inside these larger circles, contributing to the star's complexity. They are used to create the intricate patterns at each vertex of the stars. 4.
    \end{description}
\end{figure}

\begin{figure}[!ht]
    \centering
    \caption*{Sample 4: Circle with Square}
    \vspace{-1.0em}
    \scriptsize
    \setlength{\itemsep}{0pt}
    \setlength{\parskip}{0pt}
    \begin{description}
        \item[\textbf{Gemma-9b instruct}:] The image depicts several distinct shapes resembling simplified human figures. Each figure is outlined by black lines and lacks any internal details or shading. \textbf{Figure 1:} Located towards the top left, this figure has a rounded head and a slightly elongated torso. Its legs are short and stubby, ending abruptly. The arms are also relatively short and straight. \textbf{Figure 2:} Positioned to the right of Figure 1, this figure has a similar body shape but is more compact. Its head is smaller and its limbs are even shorter than those of Figure 1. \textbf{Figure 3:} Situated further down and to the right, this figure has a distinctive feature – a long, thin neck extending from the head. Its torso is somewhat wider than the previous two figures, and its limbs are proportionally longer.
        \item[\textbf{Llama-8B}:] The image depicts a chemical structure with a double bond between two carbon atoms, indicated by the straight line segment with a single bond on either side. The structure consists of four carbon atoms, with the first and last carbon atoms bound to hydrogen atoms, as indicated by the circles. The central two carbon atoms are bound to each other with a double bond, and they also have single bonds to other elements, which are not clearly visible in the image. The structure as a whole is closed, with the first and last carbon atoms connected, and the central two carbon atoms connected. The image provides a simple and visual representation of a double bond, making it easily understandable for a wide audience. The use of circles to represent hydrogen atoms is a common convention in chemical images, providing a quick and  intuitive way to indicate the presence of these elements. The straight line segments with a single bond on either side of the double bond are a clear and unambiguous  way to show the  and  bonds, making the structure easy to understand. The closure of the structure, with the first and last carbon atoms  and  the central two carbon atoms, is a critical feature  of the structure, and the image makes this clear  with the straight line segments connecting the appropriate  and.
        \item[\textbf{Gemma-9b-Quantized}:] The image depicts a complex arrangement of overlapping shapes, suggesting a fantastical landscape or cityscape. \textbf{Key Visual Elements:} \textbf{Central Figure:}  A large, rounded shape resembling a creature or character dominates the center of the composition. It has a bulbous head and a curved, elongated body that extends downwards. The details of its features are obscured by the surrounding shapes. \textbf{Interlocking Shapes:} Numerous smaller, circular, and elliptical forms weave around the central figure, creating a sense of depth and intricacy. These shapes overlap and interlock, forming a dense network. Some have slightly pointed ends, adding to the organic feel. \textbf{Color Palette:} The entire image is rendered in shades of black and white, emphasizing the interplay of light and shadow created by the overlapping shapes. This monochrome palette gives the scene a stark and dramatic quality. The overall impression is one of dynamism and complexity.
        \item[\textbf{Llama-9b-Quantized}:] An illustration showing the shape of the Earth's atmosphere.
        \item[\textbf{SPE+Qwen2-7B}:] g two black circles and a black rectangle. The circles are positioned symmetrically on either side of the rectangle, and they are identical in shape and size. The rectangle is also black and is positioned centrally between the two circles. The overall composition of the image is symmetrical, with the circles and the rectangle aligned horizontally and vertically, creating a balanced visual effect.
        \item[\textbf{SPE+Gemma-9B}:] The image depicts a single, stylized figure resembling a human form. It is drawn with a continuous, flowing line that creates a sense of movement and dynamism. The figure has a distinct elongated torso and limbs, with a somewhat abstract head shape. \textbf{Color:} The entire figure is black, creating a stark contrast against a likely white background. \textbf{Layout:} The figure is positioned centrally within the frame. It appears to be in a dynamic pose, possibly leaping or twirling. The lines that make up its body curve and bend, suggesting fluidity and energy. \textbf{Relationships:} The lack of facial features makes it difficult to interpret the figure's expression. However, the overall posture and the way the lines flow suggest a sense of grace and agility. \textbf{Overall Impression:} The image evokes a feeling of motion and elegance. Its simplicity and lack of detail allow for individual interpretation, inviting viewers to project their own thoughts and emotions onto the abstract form.
    \end{description}
    \label{fig:sample4_detail}
\end{figure}

\begin{figure}[!ht]
    \centering
    \includegraphics[width=0.25\textwidth]{images/gallery/sample_11.png}
    \caption*{Sample 11: Vintage Telephone}
    \vspace{-1.0em}
    \scriptsize
    \setlength{\itemsep}{0pt}
    \setlength{\parskip}{0pt}
    \begin{description}
        \item[\textbf{Ground Truth}:] The image depicts a vintage telephone, characterized by its classic design. The telephone is depicted in a simplified, cartoon-like style, with a focus on its main components. The main body of the telephone is cylindrical and dark blue in color, with a circular dial at the top. The dial is also dark blue and has a central hole in the middle, which is a common feature of rotary telephones. Below the dial, there is a horizontal bar that extends from the main body, which is also dark blue. This bar is connected to a smaller, circular base that is light blue with white dots around its perimeter. The base is the lowest part of the telephone and appears to be resting on a flat surface.
        \item[\textbf{BLIP-2}:] a blue microscope with a green base
        \item[\textbf{Florence-2}:] The image is an illustration of a laboratory equipment. It consists of a blue-colored microscope with a round base and a cylindrical body. The body of the microscope is made up of two columns, one on top and one on the bottom. On the top column, there is a large circular object, which appears to be a test tube or a funnel. The object is connected to a smaller sphere with small holes in it. The sphere is placed on a flat base. The background is white. The overall design is simple and minimalistic.
        \item[\textbf{Idefics3}:] The image depicts a simplified, abstract representation of a microscope. The microscope is composed of several geometric shapes. The main body is a large blue circle, which represents the base or the stand of the microscope. Above this circle, there is a smaller blue circle, likely representing the eyepiece or the objective lens. Connecting these two circles is a green horizontal bar, possibly indicating the arm or the stage of the microscope. The background is white, and there are small white dots scattered around the green bar, possibly representing dust or particles. The overall style is flat and minimalistic, with no shading or depth cues.
        \item[\textbf{BLIP-1-CPU}:] a blue and green object with white dots
        \item[\textbf{Qwen2-7B}:] The SVG illustration depicts a stylized geometric pattern composed of various colored shapes. It primarily features five distinct polygons, each with its own color scheme. \textbf{Polygon 1}: This polygon is located at the top right corner of the SVG canvas. It has four sides, two of which form a horizontal line segment that aligns with the y-axis coordinate 160. The other two sides are slanted diagonally outward from this horizontal segment to create a trapezoidal shape. The color used for this polygon is a light cyan shade (\#116,231,231). \textbf{Polygon 2}: Found near the bottom center, this polygon resembles an irregular hexagon. Its vertices are connected by curved lines and straight segments. Two of its sides lie along the x-axis at coordinates 308 and 141. The curves create a rounded appearance towards the left side of the polygon. The interior color is a dark blue shade (\#6,52,76). \textbf{Polygon 3}: Located at the top center of the SVG canvas, this polygon consists of three vertices forming an elongated triangle.
    \end{description}
\end{figure}

\begin{figure}[!ht]
    \centering
    \caption*{Sample 11: Vintage Telephone}
    \vspace{-1.0em}
    \scriptsize
    \setlength{\itemsep}{0pt}
    \setlength{\parskip}{0pt}
    \begin{description}
        \item[\textbf{Gemma-9b instruct}:] The SVG depicts a simplified, abstract representation of a microscope.

\textbf{Components:}

\textbf{Base:} A large, dark blue circle forms the base of the microscope.
\textbf{Arm/Stand:} A smaller, dark blue shape, possibly representing the arm or stand, rises from the base. It has a curved top.
\textbf{Stage:} A rectangular green bar sits horizontally across the middle, potentially representing the stage where specimens are placed. It's decorated with small white circles along its length.
\textbf{Eyepiece/Tube:} A smaller, light blue oval shape is positioned above the arm, suggesting the eyepiece or tube.
\textbf{Objective Lenses:} Two small, gray geometric shapes near the top right could represent the objective lenses.

\textbf{Style:}

The image utilizes flat colors and simple geometric shapes, creating a clean and modern aesthetic. The lack of detailed textures or shading gives it a flat design look.

\textbf{Overall Impression:}

While abstract, the arrangement of shapes and colors clearly evokes the form of a microscope, albeit in a stylized manner.
        \item[\textbf{Llama-8B}:] The image depicts a simple graphic design involving a square with rounded corners, commonly known as a rectangle with rounded edges. The square is outlined with a solid black line, which gives it a clean and professional appearance. The corners of the square are slightly rounded, which adds a soft and smooth touch to the overall shape, making it more appealing to the eye. The sides of the square are of equal length, ensuring symmetry, and the corners are well-rounded, preventing sharp edges that could be potentially dangerous. The background of the image is a plain white color, which highlights the square's characteristics and makes it stand out. The square itself is a solid black color, which provides contrast and makes its edges and corners clearly visible. The image is undated, but the design of the square suggests a contemporary approach, where rounded corners are a common feature in many products and designs, providing a more comfortable and user-friendly experience. The image is digital and appears to be a screenshot or capture from a device with a high resolution, which allows for a crisp and clear representation of the square. The image is not signed or attributed to any individual or organization, suggesting it is a generic example or a stock image available for public use.
        \item[\textbf{Gemma-9b-Quantized}:] The SVG renders a stylized image composed of several distinct geometric shapes.

\textbf{Dominant Shapes:}

\textbf{Large Blue Circle:} A prominent, dark blue circle occupies the lower central portion of the canvas. It appears to be slightly flattened at the top.
\textbf{Green Rectangle:} A bright green rectangle sits horizontally across the middle of the image, overlapping the top portion of the blue circle. It has a series of small white dots running along its length.
\textbf{Smaller Blue Shape:} To the left of the green rectangle, there's a smaller, teardrop-shaped element in a lighter shade of blue. It points downwards and slightly inwards towards the center.
\textbf{Two Grey Shapes:} In the upper right corner, two grey shapes resemble stylized rocks or perhaps abstract clouds. They are irregular in form and positioned close together.

\textbf{Composition:}

The shapes are arranged in a somewhat loose and asymmetrical manner, creating a sense of movement and dynamism. The bright green rectangle acts as a visual anchor, contrasting with the cooler blue tones of the circle and teardrop.
        \item[\textbf{Llama-9b-Quantized}:] The SVG code defines a simple drawing of a microscope. It is composed of basic shapes like circles and rectangles, colored in shades of blue, green, and grey.
    \end{description}
\end{figure}

\begin{figure}[!ht]
    \centering
    \caption*{Sample 11: Vintage Telephone (continued)}
    \vspace{-1.0em}
    \scriptsize
    \setlength{\itemsep}{0pt}
    \setlength{\parskip}{0pt}
    \begin{description}
        \item[\textbf{SPE+Qwen2-7B}:] The image depicts a simple, abstract design consisting of two primary shapes: a circle and a rectangle. The circle is positioned above the rectangle, and both shapes are colored in a gradient pattern. The circle is filled with a teal color that transitions from a darker shade at the top to a lighter shade towards the bottom. The rectangle below the circle is also filled with a similar gradient pattern, but it has a slightly different hue. It starts with a darker shade at the left end and transitions to a lighter shade towards the right end. The circle and rectangle are aligned vertically, with the circle placed directly above the rectangle. The shapes appear to be symmetrical, with the circle's center aligned directly above the rectangle's center. There is no text or additional elements present in the image, making it a minimalist composition focused solely on the interaction between the two shapes.
        \item[\textbf{SPE+Gemma-9B}:] The image depicts a scene with several distinct shapes and colors. \textbf{Dominant Shapes:} \textbf{Large Blue Circle:} The largest element is a large blue circle located towards the bottom center of the image. This circle appears to be filled with a deep teal color. It has a slight curve at its top suggesting it might be part of a larger structure or object. \textbf{Green Rectangle:} A bright green rectangle sits horizontally across the middle of the image, slightly overlapping the blue circle. This rectangle seems to have rounded corners. \textbf{White Dots:} Scattered around the green rectangle are small white dots arranged in a somewhat symmetrical pattern. These dots seem to be placed intentionally, possibly representing some kind of decorative element or function within the image. \textbf{Smaller Blue Shape:} Towards the left side of the image, there's a smaller blue shape resembling a stretched oval or a teardrop. It has a smooth, curved form and sits slightly above the green rectangle. \textbf{Purple/Gray Shapes:} There are two smaller, purple/gray shapes positioned near the top right corner of the image. They appear to be more abstract and geometric in nature, with sharp angles and lines. \textbf{Color Palette:} \textbf{Blue:} Predominantly used for the large circular elements, anchoring the composition. \textbf{Green:} Highlights the central rectangle and contrasts with the surrounding blues. \textbf{White:} Appears as small dots that add visual rhythm around the rectangle. \textbf{Purple/Gray:} Provide secondary accents near the top right of the scene.
    \end{description}
    \label{fig:sample11_detail}
\end{figure}

\begin{figure}[!ht]
    \centering
    \includegraphics[width=0.25\textwidth]{images/gallery/sample_13.png}
    \caption*{Sample 13: 50 Sign}
    \vspace{-1.0em}
    \scriptsize
    \setlength{\itemsep}{0pt}
    \setlength{\parskip}{0pt}
    \begin{description}
        \item[\textbf{Ground Truth}:] The image depicts the number 50 is written on the black color object.
        \item[\textbf{BLIP-2}:] 50 km/h speed limit sign
        \item[\textbf{Florence-2}:] The image is a black and white circular sign with the number 50 in the center. The number 50 is written in a bold, sans-serif font and is surrounded by a thin, black line. The line is crossed out in a diagonal direction, indicating that the number is not allowed. The sign appears to be a prohibition or restriction symbol.
        \item[\textbf{Idefics3}:] The image depicts a round traffic sign with a black border and a white background. The sign features a large, bold number "50" in black, occupying the central portion of the sign. The number "50" is prominently displayed, with its height and width making it easily readable from a distance. The sign also includes a black diagonal line extending from the upper left corner to the lower right corner, forming an "X" shape. This diagonal line is positioned in such a way that it intersects the number "50" and crosses through the center of the sign.
        \item[\textbf{BLIP-1-CPU}:] a black and white sign with the number 50
        \item[\textbf{Qwen2-7B}:] The SVG code describes a complex, symmetrical design composed of several geometric shapes. It features four main sections: \textbf{Circular Arcs:} Two sets of circular arcs create overlapping rings with alternating white and black sections. Each set has two concentric circles that gradually expand outwards, forming a ring-like structure. - White sections: Circles start at point (256, 511) and grow outward. - Black sections: Circles begin at point (436, 436), creating a second, larger set of overlapping rings. \textbf{Curved Lines and Polygons:} These elements form intricate patterns within the central area of the image. They include: - Curved lines connecting various points to create a polygon shape near the center. - Another curved line segment starting from point (228, 195) to point (162, 238). \textbf{Triangle Pattern:} Located towards the right side of the SVG, this pattern comprises multiple small triangles connected by their vertices.
    \end{description}
\end{figure}

\begin{figure}[!ht]
    \centering
    \caption*{Sample 13: 50 Sign}
    \vspace{-1.0em}
    \scriptsize
    \setlength{\itemsep}{0pt}
    \setlength{\parskip}{0pt}
    \begin{description}
        \item[\textbf{Gemma-9b instruct}:] The image depicts a stylized black and white representation of a human eye. The iris is depicted as a large, rounded shape with concentric circles radiating outward from its center. These circles are interrupted by smaller, circular pupils within the larger iris. The pupil appears to be partially obscured by a thin, curved line that suggests a tear rolling down the cheek. The surrounding sclera, or white part of the eye, is represented by a series of smooth, flowing curves that create a sense of depth and dimension. The overall effect is one of both fragility and strength, highlighting the complexity and beauty of the human eye.
        \item[\textbf{Llama-8B}:] The image depicts a simple geometric design consisting of a square with rounded corners, positioned at the center of the image. The square is outlined by a solid black line, and it is uniform in color, suggesting a white or transparent background. The corners of the square are slightly rounded, giving it a softer and more rounded appearance compared to a traditional square with sharp corners. The sides of the square are of equal length, and the overall shape is symmetrical, with the diagonal lines bisecting the square into four identical right triangles. The corners of the square are rounded such that the vertices are slightly rounded, and the sides are straight and of equal length, making the overall shape symmetrical. The square has a solid black outline, and its interior is empty, leaving it appear white or transparent. The corners of the square are rounded, and the sides are straight and of equal length, making the overall shape symmetrical. The square has a solid black outline, and its interior is empty, leaving it appear white or transparent. The corners of the square are rounded, and the sides are straight and of equal length, making the overall shape symmetrical. The square has a solid black outline, and its interior is empty, leaving it appear white or transparent.
        \item[\textbf{Gemma-9b-Quantized}:] The SVG renders a complex scene depicting a three-dimensional sphere nestled within a larger circle. \textbf{Sphere:} The sphere itself is predominantly white and is formed by a series of smooth curves that transition seamlessly into one another. It occupies the central portion of the image, appearing to be slightly tilted towards the viewer. \textbf{Outer Circle:} Encasing the sphere is a dark gray circle. This circle acts as a backdrop and provides a clear contrast against the whiteness of the sphere. \textbf{Intersecting Lines:} Numerous intersecting lines emanate from the center of the outer circle, creating a network of pathways that seem to converge and diverge around the sphere. These lines are also black and add a sense of dynamism and complexity to the composition. \textbf{Additional Shapes:} Within the space between the sphere and the outer circle, there are several smaller shapes. Some appear to be abstract forms, while others resemble stylized leaves or petals. These shapes are also black and contribute to the intricate and multifaceted nature of the design.
        \item[\textbf{Llama-9b-Quantized}:] A technical writer provides a precise description of an SVG scene.
    \end{description}
\end{figure}

\begin{figure}[!ht]
    \centering
    \caption*{Sample 13: 50 Sign (continued)}
    \vspace{-1.0em}
    \scriptsize
    \setlength{\itemsep}{0pt}
    \setlength{\parskip}{0pt}
    \begin{description}
        \item[\textbf{SPE+Qwen2-7B}:] The image depicts a simple geometric shape, specifically a circle divided into two equal parts by a straight line that runs from the top left to the bottom right. The circle is black, and the line is also black, creating a clear contrast against the white background. The division of the circle is symmetrical, meaning that each half of the circle is identical in size and shape. The circle itself is a closed loop without any additional features or decorations.
        \item[\textbf{SPE+Gemma-9B}:] The image depicts a stylized black and white representation of a face within a larger circular frame. \textbf{Central Features:} \textbf{Eyes:} Two large, oval shapes positioned symmetrically above the center line of the circle represent the eyes. They have a slightly exaggerated, cartoonish appearance. \textbf{Nose:} A small, curved triangle situated between the eyes suggests the nose. \textbf{Mouth:} Below the nose, a gentle curve forms the mouth, hinting at a smiling expression. \textbf{Frame:} The entire composition is contained within a circular border, which appears to be thicker than the central features. This frame emphasizes the isolation and prominence of the face. \textbf{Style:} \textbf{Black and White:} The image utilizes a stark contrast between black and white, creating a graphic and minimalist aesthetic. \textbf{Geometric Shapes:} The face is constructed primarily from geometric shapes like circles, ovals, and triangles, contributing to its simplified and abstract nature. \textbf{Overall Impression:} The image evokes a sense of playfulness and innocence due to the rounded shapes and the gentle smile. The simplicity of the design makes it easily recognizable as a face, while the lack of detail allows for individual interpretation and imagination. It could be interpreted as a symbol of joy and unity.
    \end{description}
    \label{fig:sample13_detail}
\end{figure}

\begin{figure}[!ht]
    \centering
    \includegraphics[width=0.25\textwidth]{images/gallery/sample_16.png}
    \caption*{Sample 16: Shopping Cart}
    \vspace{-1.0em}
    \scriptsize
    \setlength{\itemsep}{0pt}
    \setlength{\parskip}{0pt}
    \begin{description}
        \item[\textbf{Ground Truth}:] The image depicts a piece of luggage, specifically a suitcase. The suitcase is designed to be portable and is equipped with wheels, which are visible at the bottom of the image. The suitcase has a handle at the top, which is likely used for carrying the luggage. The handle is positioned at the top center of the suitcase, extending upwards and slightly backwards, giving it a handle-like appearance. The suitcase is rectangular in shape and has a flat top surface, which is typical for luggage designed to be easily stacked or placed on top of other luggage.
        \item[\textbf{BLIP-2}:] a black and white illustration of a suitcase
        \item[\textbf{Florence-2}:] The image is a black and white icon of a shopping cart. The cart has a rectangular shape with a handle on top and four wheels at the bottom. The handle is curved and appears to be made of metal. The body of the cart is rectangular in shape and has a small opening on the front. The wheels are also curved and have a small hole in the center. The background is plain white.
        \item[\textbf{Idefics3}:] The image depicts a suitcase with a handle and wheels. The suitcase is upright, with the handle extended upwards. The wheels are visible at the bottom of the suitcase, indicating that it is designed for mobility. The suitcase appears to be closed, with no visible details of its contents or exterior design. The image is minimalist, focusing solely on the essential features of the suitcase.

        \begin{itemize}
            \item \textbf{Handle and Wheels}: The presence of a handle and wheels suggests that the suitcase is designed for ease of transport, making it suitable for travel. The handle allows for easy pulling, while the wheels facilitate smooth movement over various surfaces.
        \end{itemize}
        \item[\textbf{BLIP-1-CPU}:] shopping cart icon vector
    \end{description}
\end{figure}

\begin{figure}[!ht]
    \centering
    \caption*{Sample 16: Shopping Cart}
    \vspace{-1.0em}
    \scriptsize
    \setlength{\itemsep}{0pt}
    \setlength{\parskip}{0pt}
    \begin{description}
        \item[\textbf{Qwen2-7B}:] The SVG code describes two distinct shapes drawn using paths within its structure.

        Firstly, there is a smooth curve starting at point (447, 448) that moves towards point (434, 435), then it smoothly transitions to point (434, 411) and continues until it reaches point (459, 435). This creates an ellipse-like shape with a slight curvature. The color used for this shape is black ('fill:0,0,0') without any stroke outline ('stroke:None'). It appears to be a symmetrical figure with its center at approximately (447, 448).

        Secondly, another path starts at point (512, 421) and extends horizontally to point (479, 158), forming a horizontal line segment which has been bent into an arc shape resembling a large semi-circle. The endpoints of this arc meet at point (59, 44). The fill color for this area remains black, but it's filled with a white border ('stroke:None'), making the border invisible. The opacity is set to 1 indicating full visibility.
        \item[\textbf{Gemma-9b instruct}:] The image depicts a stylized human figure, rendered in black against a white background.  The figure is composed of several distinct shapes:
        \begin{itemize}
            \item \textbf{Torso:} A large, rounded rectangle forms the central part of the body.
            \item \textbf{Head:}  A smaller oval shape sits atop the torso.
            \item \textbf{Arms:} Two elongated, slightly curved lines extend from either side of the torso.
            \item \textbf{Legs:}  Two similar lines, slightly shorter than the arms, emerge from the bottom of the torso.
        \end{itemize}
        The figure's proportions are exaggerated, with long limbs and a small head. The overall impression is one of simplicity and abstraction, focusing on basic geometric shapes to represent the human form.

        There are also two additional elements within the image:
        \begin{itemize}
            \item \textbf{Left Hand:} A smaller, rectangular shape positioned at the end of the left arm.
            \item \textbf{Right Foot:} A circular shape located at the end of the right leg.
        \end{itemize}
    \end{description}
\end{figure}

\begin{figure}[!ht]
    \centering
    \caption*{Sample 16: Shopping Cart (continued)}
    \vspace{-1.0em}
    \scriptsize
    \setlength{\itemsep}{0pt}
    \setlength{\parskip}{0pt}
    \begin{description}
        \item[\textbf{Llama-8B}:] The image depicts a stylized representation of a lightning bolt. The lightning bolt is depicted in a two-dimensional, flat style, with the structure consisting of a straight vertical line segment, a curve to the right and then back to the left, and another straight vertical line segment. The lightning bolt has a smooth, continuous curve, with no sharp edges or corners. The base of the lightning bolt, where it meets the vertical line segments, is slightly wider than the top, giving it a slight taper. The overall shape is symmetrical, with the curve on one side being a mirror image of the curve on the other side. The lightning bolt is outlined with a solid black line, which provides a clear distinction between the object and its background, making it stand out prominently. The underlying surface is also black, which complements the outline, making the object easily recognizable. The image presents a minimalistic approach, with only the essential elements included, such as the straight lines and curves that make up the structure of the lightning bolt. There are no additional details or decorations, keeping the focus on the simplicity and elegance of the design. The lightning bolt is depicted against a black background, which allows it to shine through, making it the primary visual element of the image.
        \item[\textbf{Gemma-9b-Quantized}:] The SVG renders a complex, detailed drawing of a stylized bird in flight. The bird's body is depicted as a flowing, rounded shape that appears almost like a teardrop, with a long, slender tail trailing behind it.

        The bird's wings are outstretched, forming broad, curved shapes that suggest powerful motion. Its head is small and round, tucked into its chest as if preparing to dive. There's a subtle suggestion of a beak near the front of the bird's head.

        The entire bird is rendered in solid black, giving it a bold and dramatic presence against a hypothetical white background.  There are no gradients or textures applied to the bird, emphasizing its simple yet elegant form.

        Beneath the bird, there's a smaller, more abstract element. It resembles a series of interconnected curves and loops, possibly representing swirling air currents or the bird's wake. This element also uses the same solid black fill as the bird.

        The overall composition suggests a dynamic scene of a bird soaring through the air, leaving a trail of disturbed energy behind it.
        \item[\textbf{Llama-9b-Quantized}:] A simple scene with two lines and four curves.
    \end{description}
\end{figure}

\begin{figure}[!ht]
    \centering
    \caption*{Sample 16: Shopping Cart (continued)}
    \vspace{-1.0em}
    \scriptsize
    \setlength{\itemsep}{0pt}
    \setlength{\parskip}{0pt}
    \begin{description}
        \item[\textbf{SPE+Qwen2-7B}:] The image depicts a simple, single symbol consisting of two distinct parts. The symbol is enclosed within a square frame. The upper part of the symbol features a horizontal line that extends across the top of the square, while the lower part consists of a vertical line that extends from the bottom of the square. Both lines intersect at the center of the square, forming an "X" shape. The entire symbol is black, and the background is white, making the symbol stand out clearly against the background.
        \item[\textbf{SPE+Gemma-9B}:] The image depicts a stylized human figure composed of interconnected geometric shapes.

        \textbf{Body:} The main body is formed by a series of curved lines that suggest a rounded torso and limbs. The upper body has a more elongated shape, hinting at shoulders and a head.

        \textbf{Head:}  A distinct circular shape at the top represents the head. It's slightly offset to the right, giving the figure a dynamic pose.

        \textbf{Legs:}  Two downward-pointing triangles represent the legs, connected to the torso by a wider, curved line.

        \textbf{Arms:}  Two shorter, narrower triangles extending from the sides of the torso suggest arms. They lack defined hands or wrists.

        \textbf{Details:} There are several small, rectangular shapes scattered around the figure, possibly representing facial features or accessories. However, due to their simplicity, it's difficult to discern their exact purpose.

        \textbf{Color:} The entire figure is black against a white background, creating a strong contrast that emphasizes its form.

        \textbf{Layout:} The figure is positioned centrally within the frame, with no other elements present. This focus on the single figure creates a minimalist and abstract composition.

        \textbf{Overall Impression:}  The image evokes a sense of movement and dynamism through the arrangement of the shapes. The simplicity of the design keeps the focus on the interplay between the geometric forms.
    \end{description}
    \label{fig:sample16_detail}
\end{figure}


\begin{figure}[!ht]
    \centering
    \includegraphics[width=0.25\textwidth]{images/gallery/sample_24.png}
    \caption*{Sample 24: Shield Icon}
    \vspace{-1.0em}
    \scriptsize
    \setlength{\itemsep}{0pt}
    \setlength{\parskip}{0pt}
    \begin{description}
        \item[\textbf{Ground Truth}:] The image depicts a simple, abstract design that resembles a shield or a box with a grid pattern. The design consists of a central square shape with four smaller squares arranged symmetrically around it. Each of these smaller squares is divided into two equal parts by a horizontal and a vertical line, creating a total of eight smaller rectangles within the larger square.
        \item[\textbf{BLIP-2}:] a black and white logo of a shield with wings
        \item[\textbf{Florence-2}:] The image is a black and white line drawing of a shield or shield with four panels. The shield is divided into four sections, each with a square in the center. The panels are arranged in a grid-like pattern, with each panel having a small square in between them. On either side of the shield, there are two curved lines that form the shape of a cup or mug. The overall design is simple and minimalistic.
        \item[\textbf{Idefics3}:] The image is a minimalist, two-dimensional representation of a shield. The shield is depicted in a stylized, geometric form, with a symmetrical design. The shield is divided into four quadrants, each containing a square grid pattern. The grid pattern within each quadrant consists of small, evenly spaced squares. The shield's outline is simple and clean, with straight lines forming the edges of the shield. The overall shape of the shield is a perfect square, with equal sides and angles. The shield is positioned centrally within the image, with no background or additional elements present.
        \item[\textbf{BLIP-1-CPU}:] a black and white logo with a large, rectangular shape
    \end{description}
\end{figure}

\begin{figure}[!ht]
    \centering
    \caption*{Sample 24: Shield Icon}
    \vspace{-1.0em}
    \scriptsize
    \setlength{\itemsep}{0pt}
    \setlength{\parskip}{0pt}
    \begin{description}
        \item[\textbf{Qwen2-7B}:] The SVG code describes a complex polygon with multiple segments forming intricate shapes within a 2D plane.

        The primary structure is a large pentagon made up of several interconnected lines and arcs. It starts with a smaller pentagon at its center, followed by outer layers that expand outward symmetrically. This central pentagon has sides that are slightly curved towards each other, creating a star-like appearance when viewed from afar.

        Adjacent to this central pentagon, there's a larger pentagonal shape formed by extending the original edges outward. The outer edges have slight curves as well, making the overall figure look like a series of overlapping pentagons.

        In between these two pentagonal structures, there are additional elements resembling triangles and trapezoids. These forms are created using straight lines and arcs connecting various points around the central pentagon.

        A key feature is the presence of a thin strip along the right side of the main pentagonal shape, which consists of horizontal lines and curved segments. This strip adds depth and complexity to the design.

        Throughout the entire drawing, the color is black (`\#000000`), giving the shapes a crisp and solid appearance.
        \item[\textbf{Gemma-9b instruct}:] The image depicts a complex cityscape with distinct architectural elements.  A prominent feature is a large, rectangular building at the center, resembling a modern office tower. It has multiple floors represented by horizontal lines and appears to have a slight upward slant towards its top.

        Flanking this central structure are smaller buildings, each with unique designs. Some resemble narrow towers while others appear more like elongated rectangles. These buildings are scattered across the image, creating a sense of depth and a bustling urban environment.

        Numerous windows punctuate the facades of these buildings, adding visual interest and suggesting activity within. The windows are mostly uniform in size and shape, contributing to a somewhat ordered aesthetic.

        Interspersed among the buildings are stylized trees, depicted as simple triangular shapes with rounded tops. They add a touch of greenery and contrast against the stark lines of the cityscape.

        Overall, the image conveys a dynamic urban scene, filled with intricate details and a sense of movement and energy.
        \item[\textbf{Llama-8B}:] The image depicts a simple line drawing of a speech bubble. The speech bubble is outlined with a solid black line, and it has a slightly rounded triangular shape with a broad base and a sharp point at the top. The inside of the speech bubble is empty, indicating that no actual words or text are present. The outline of the speech bubble is smooth and continuous, with no visible gaps or irregularities. The background of the image is also black, which makes the speech bubble stand out prominently.

        Explanation: The speech bubble depicted in this image is designed for visual representation and understanding, rather than for the transmission of specific verbal content. The bubble's structure suggests that it is intended to contain words or a message, but the interior is empty, providing a blank canvas for additional information.
        \item[\textbf{Gemma-9b-Quantized}:] The SVG depicts a whimsical scene reminiscent of a medieval tapestry or a child's drawing.  A large central figure dominates the composition. This figure appears to be a knight or warrior, clad in armor and wielding a sword. Its body is formed by a series of connected curved lines, suggesting movement and dynamism.

        Surrounding the central figure is a complex network of smaller shapes and figures. Some resemble houses, others trees, and some seem like fantastical creatures. These elements are arranged in no particular order, creating a sense of chaos and vibrancy. The overall impression is one of a bustling marketplace or a mythical realm.

        The color palette is predominantly black and white, with subtle hints of grey. This monochrome scheme lends a timeless quality to the image, further emphasizing its archaic feel.
        \item[\textbf{Llama-9b-Quantized}:] The scene shows an example of the intersection between two circles.
    \end{description}
\end{figure}

\begin{figure}[!ht]
    \centering
    \caption*{Sample 24: Shield Icon (continued)}
    \vspace{-1.0em}
    \scriptsize
    \setlength{\itemsep}{0pt}
    \setlength{\parskip}{0pt}
    \begin{description}
        \item[\textbf{SPE+Qwen2-7B}:] The image depicts a simple geometric figure consisting of two main components: a large triangle and a smaller triangle inside it. The larger triangle is positioned above the smaller one. Both triangles are outlined by straight lines, and they are filled with black color. The larger triangle has three sides, while the smaller triangle has two sides that are part of the larger triangle's base.

        The larger triangle is oriented with its base pointing upwards, and the smaller triangle is positioned below it, sharing part of the base of the larger triangle. The smaller triangle is also oriented with its base pointing downwards, creating a V-shape within the larger triangle. The two triangles share a common vertex at the top of the larger triangle, where the smaller triangle begins.

        The overall structure resembles a double triangle, with the smaller triangle nested inside the larger one. The lines forming the triangles are straight and sharp, indicating a clean, precise design. The colors used are monochromatic, with both triangles colored in black, which makes them stand out against any background.
        \item[\textbf{SPE+Gemma-9B}:] The image depicts a stylized city scene composed of numerous rectangular buildings arranged in a somewhat chaotic yet organized manner.

        \textbf{Buildings:}
        \begin{itemize}
            \item The buildings are primarily represented by rectangles of varying sizes and orientations. Some are taller than others, creating a sense of depth and variation.
            \item Many buildings have slightly rounded edges, giving them a softer, more organic appearance.
            \item There is a clear emphasis on rooftops, which are often depicted with distinctive shapes. Some feature curved lines that suggest chimneys or other architectural details.
        \end{itemize}

        \textbf{Layout and Composition:}
        \begin{itemize}
            \item The buildings cluster together, forming distinct blocks within the cityscape.
            \item While there's no discernible grid pattern, the arrangement suggests a bustling urban environment.
            \item There's a sense of movement implied through the angles and directions of the building outlines.
        \end{itemize}

        \textbf{Color:}
        \begin{itemize}
            \item The entire image is rendered in black and white, emphasizing the geometric forms and shadows cast by the buildings.
        \end{itemize}

        \textbf{Other Details:}
        \begin{itemize}
            \item There are several smaller, triangular shapes scattered throughout the composition, possibly representing trees or streetlamps. These add visual interest and detail to the cityscape.
            \item At the bottom center of the image, there's a wider, horizontal rectangle that could represent a road or a river.
        \end{itemize}
    \end{description}
    \label{fig:sample24_detail}
\end{figure}

\begin{figure}[!ht]
    \centering
    \includegraphics[width=0.25\textwidth]{images/gallery/sample_25.png}
    \caption*{Sample 25: Tablet with Dollar}
    \vspace{-1.0em}
    \scriptsize
    \setlength{\itemsep}{0pt}
    \setlength{\parskip}{0pt}
    \begin{description}
        \item[\textbf{Ground Truth}:] The image depicts a stylized, minimalist design of a tablet. The tablet is held in a hand, with the hand and tablet forming a central focus of the image. The tablet is black, and the hand holding it is also black, suggesting a monochromatic color scheme. The tablet's screen displays a dollar sign (\$) which is white and positioned centrally.
        \item[\textbf{BLIP-2}:] a hand holding a tablet with a dollar sign on it
        \item[\textbf{Florence-2}:] The image is a black and white icon of a tablet with a dollar sign on the screen. The tablet is held in a hand, with the fingers slightly curled around the edges of the screen, as if it is being held by someone's hand. The dollar sign is in the center of the tablet, with a white background and a black outline. The image is simple and minimalistic, with no other elements or design elements.
        \item[\textbf{Idefics3}:] The image depicts a tablet with a black border and a white screen. On the screen, there is a dollar sign (\$), which is white and positioned centrally. The tablet is held by a hand, which is also black. The hand is positioned in the lower right corner of the image, with the tablet resting in the palm of the hand. The background of the image is white, providing a stark contrast to the black tablet and hand. The image is simple and minimalistic, focusing on the tablet and the dollar sign without any additional elements or distractions.
        \item[\textbf{BLIP-1-CPU}:] a hand holding a tablet with a dollar sign on it
    \end{description}
\end{figure}

\begin{figure}[!ht]
    \centering
    \caption*{Sample 25: Tablet with Dollar}
    \vspace{-1.0em}
    \scriptsize
    \setlength{\itemsep}{0pt}
    \setlength{\parskip}{0pt}
    \begin{description}
        \item[\textbf{Qwen2-7B}:] The SVG rendering depicts a complex geometric pattern. A series of interconnected lines create multiple shapes within the canvas, forming various enclosed spaces that vary in size and orientation.

        At the top left corner, there is a smaller rectangle with rounded corners. It has dimensions approximately 6 units wide by 4 units high. Its edges are slightly curved at the corners, giving it a soft appearance. This shape is positioned along the edge of the larger canvas area.

        Moving to the right side, a larger rectangular shape appears, extending horizontally across about half of the canvas width. This rectangle also features rounded corners but with more pronounced curvature compared to the previous one. Its dimensions are roughly twice as large, measuring around 12 units wide by 4 units tall.

        Below these rectangles, a cluster of smaller triangles emerges. These triangles are positioned closer together, creating a dense arrangement. Each triangle shares its base with another, forming a continuous line. The height of each triangle varies slightly, adding a dynamic element to this section.

        A zigzag-like structure extends from the bottom portion of the larger rectangle towards the bottom part of the canvas. This zigzag consists of sharp angles and varying lengths, contributing to a sense of movement and irregularity.
        \item[\textbf{Gemma-9b instruct}:] The SVG depicts a stylized human figure within a larger geometric composition.

        \textbf{Figure:}
        \begin{itemize}
            \item \textbf{Body:} The figure is primarily represented by a series of connected curves and lines forming a simplified silhouette. It has a rounded head, a slightly elongated torso, and narrow legs.
            \item \textbf{Arms:}  The arms are depicted as straight lines extending from the sides of the torso. They appear to be raised slightly.
            \item \textbf{Head:} The head lacks detailed features, but its shape suggests a rounded form. There's a slight curve at the top that could represent hair.
            \item \textbf{Legs:} The legs are relatively short and thin, ending in small points.
        \end{itemize}

        \textbf{Geometric Background:}
        \begin{itemize}
            \item \textbf{Rectangle:} A large rectangle spans the bottom portion of the image, creating a sense of depth and grounding the figure.
            \item \textbf{Curves:} Several curved lines intersect with the rectangle, forming a complex pattern that adds visual interest. These curves also interact with the figure, creating a dynamic interplay between the organic shape of the body and the geometric background.
        \end{itemize}

        \textbf{Colors:}
        \begin{itemize}
            \item The entire image is rendered in shades of black and white. This creates a stark contrast that highlights the shapes and forms.
        \end{itemize}

        Overall, the image evokes a minimalist aesthetic.
    \end{description}
\end{figure}

\begin{figure}[!ht]
    \centering
    \caption*{Sample 25: Tablet with Dollar (continued)}
    \vspace{-1.0em}
    \scriptsize
    \setlength{\itemsep}{0pt}
    \setlength{\parskip}{0pt}
    \begin{description}
        \item[\textbf{Llama-8B}:] The image depicts a symbol that is commonly associated with privacy and security. The symbol consists of a circle with a solid line intersecting it at two points, forming a segment of a circle. The circle is enclosed within a square, which is also a common emblem for privacy and security. The circle-squared structure is often used to represent a system or device that is designed to protect personal information or ensure security, such as a password manager, an encrypted communication app, or a biometric authentication tool. The circle in this symbol is typically black, indicating confidence or certainty, while the square is usually white, suggesting neutrality or objectivity. The solid line that connects the circle to the square is straight and smooth, implying reliability or trustworthiness. The segment of a circle formed by the line is symmetrical, suggesting balance or fairness, while the entire circle is complete, suggesting wholeness or integrity. The circle-squared symbol is often used to assure individuals that their personal data is safe and secure, and that they can trust the system or device they are using to protect their privacy. The symbol is also a visual representation of the idea that security and privacy go hand in hand, and that both are essential for a healthy digital lifestyle.
        \item[\textbf{Gemma-9b-Quantized}:] The SVG renders a complex scene depicting a stylized bird in flight against a dark background.

        The bird's body is predominantly black, forming a teardrop shape that extends upwards towards the tail. The head is more rounded and features a small beak at its tip. Feathered details are subtly suggested by lines and curves across the bird's wings and body. The wings themselves are outstretched and angled downwards, creating a sense of dynamic movement.

        Beneath the bird, there's a smaller, abstract element resembling a swirling cloud or plume. This element is also primarily black but has lighter shades interspersed within it, suggesting texture and depth. A thin white line outlines the bottom edge of this element, adding further definition.

        Adding to the complexity, the composition includes a series of overlapping geometric shapes in muted tones of gray and brown. These shapes create a subtle sense of landscape or terrain beneath the bird and cloud.
        \item[\textbf{Llama-9b-Quantized}:] The 3D Scene
    \end{description}
\end{figure}

\begin{figure}[!ht]
    \centering
    \caption*{Sample 25: Tablet with Dollar (continued)}
    \vspace{-1.0em}
    \scriptsize
    \setlength{\itemsep}{0pt}
    \setlength{\parskip}{0pt}
    \begin{description}
        \item[\textbf{SPE+Qwen2-7B}:] The image depicts a simple, minimalistic design featuring two black circles arranged vertically against a white background. The circles are positioned side by side, with one slightly taller than the other. The larger circle is located above the smaller one. Both circles have a uniform thickness and are filled with a solid black color. The smaller circle is placed directly below the larger one, creating a slight overlap where they touch. The overall composition of the image is symmetrical, with the vertical alignment of the circles being the focal point.
        \item[\textbf{SPE+Gemma-9B}:] The image depicts a stylized human figure with several distinct elements.

        \textbf{Figure:}
        \begin{itemize}
            \item The central figure is depicted in a somewhat abstract manner. Its torso is represented by a thick, black rectangle with rounded edges.
            \item Two smaller rectangles extending from this torso suggest arms, though they are simplified and lack hands.
            \item At the top of the torso, another rectangle forms the head, also lacking detailed features like eyes or a nose.
        \end{itemize}

        \textbf{Additional Elements:}
        \begin{itemize}
            \item A series of curved lines emanating from the bottom of the torso represent legs, flowing outwards and then curving upwards slightly before disappearing. These lines create a sense of movement or fluidity.
            \item \textbf{Geometric Shapes:} Scattered around the figure are various geometric shapes – circles, squares, and triangles – in black. These seem to act as decorative elements or possibly symbolic representations.
            \item \textbf{Background:} The background is white, making the black figure and its accompanying shapes stand out prominently.
        \end{itemize}

        \textbf{Overall Impression:} The image conveys a sense of minimalism and abstraction. The figure lacks realistic details, relying instead on simple shapes to convey a basic human form. The scattered geometric shapes add a touch of complexity and intrigue, inviting viewers to interpret their meaning. The overall impression is one of simplicity, yet there is enough variety to keep the composition engaging.
    \end{description}
    \label{fig:sample25_detail}
\end{figure}

\begin{figure}[!ht]
    \centering
    \includegraphics[width=0.25\textwidth]{images/gallery/sample_21.png}
    \caption*{Sample 21: Import Requests}
    \vspace{-1.0em}
    \scriptsize
    \setlength{\itemsep}{0pt}
    \setlength{\parskip}{0pt}
    \begin{description}
        \item[\textbf{Ground Truth}:] The image depicts import requests
        \item[\textbf{BLIP-2}:] a flat icon of a bar chart on a page
        \item[\textbf{Florence-2}:] The image is a graphic design of a bar graph. It is a rectangular icon with a red background and a white border. The bar graph is divided into four sections, each with a different color - blue, green, and white. The blue section is on the left side of the icon, the green section is in the middle, and the white section is at the bottom. The bars are arranged in a horizontal line, with the blue section on the right side and the green and white sections on the bottom left side. The overall design is simple and minimalistic.
        \item[\textbf{Idefics3}:] The image depicts a graphical representation of a bar chart. The chart consists of two main sections: a horizontal bar chart and a vertical bar chart. \textbf{Horizontal Bar Chart:} The horizontal bar chart is positioned at the top of the image. It features two bars, each colored in red and gray. The red bar is shorter than the gray bar. The horizontal bar chart is likely representing data across different categories, with the red bar indicating a smaller value and the gray bar indicating a larger value.
        \item[\textbf{BLIP-1-CPU}:] a red and white icon with a bar chart
        \item[\textbf{Qwen2-7B}:] A symmetrical architectural design consisting of two identical sections separated by a narrow vertical space. The main structure is composed of five stacked horizontal rectangles with different shades of grey, alternating between a lighter grey and a darker grey. The upper section features two thinner rectangles slightly shifted towards each other horizontally, forming a V-shape along their top edges when viewed from above. This creates a small triangular space at the top that appears to be cut out but is filled with a bright red color representing a door. The middle section contains four rectangles stacked vertically with the same light and dark grey alternation as the outer structure. At the bottom right corner of this section, there's a smaller rectangle attached to the larger one, giving it an additional shape resembling a rounded corner. Below these two sections lies a series of three rectangles that extend further down, with the leftmost and rightmost rectangles having rounded corners, creating a subtle arch-like appearance across the entire bottom part of the design. Each rectangle has thin white borders outlining them, providing clear separation and definition within the design. The entire drawing appears to have a uniform style throughout its elements.
    \end{description}
\end{figure}

\begin{figure}[!ht]
    \centering
    \caption*{Sample 21: Import Requests}
    \vspace{-1.0em}
    \scriptsize
    \setlength{\itemsep}{0pt}
    \setlength{\parskip}{0pt}
    \begin{description}
        \item[\textbf{Llama-8B}:] The image depicts a simple line drawing of a dog with a black coat. The dog is positioned in a standing position, with its front legs straight and its back legs bent slightly, giving it a crouching appearance. The dog has a long snout, rounded nose, and prominent lips, characteristic of many breed of dogs. Its eyes are depicted as circular shapes, set well apart, and its ears are long and pointed, tapering at the tips, a feature common in several dog breeds. The dog's coat is solid black, covering its body evenly, and its paws are dark in color, matching its fur, suggesting nail polish or other coating. The background of the image is white, providing a clean and bright appearance to the picture. The lines used to draw the dog are thin and smooth, giving it a sleek and symmetrical look, while still maintaining a distinct and recognizable shape. The image is unscaled, which means that the dimensions of the picture are not altered, preserving the original size and proportion of the elements. The canvas on which the image is drawn is rectangular, with a smooth edge, and the pixels are uniformly spaced, ensuring a sharp and clear visual representation.
        \item[\textbf{Gemma-9b instruct}:] The SVG renders a graphic of a stylized smartphone. The main body of the phone is represented by a large, rectangular shape filled with a light grey color. This rectangle extends from the left to right edge of the image, forming the top portion of the phone's screen. Below this rectangular area, there is a smaller, solid red rectangle positioned at the bottom right corner. This likely represents the phone's battery indicator. On the upper part of the phone, within the larger light grey rectangle, there are several horizontal lines with a dark blue fill. These lines suggest navigation bars or app icons on the phone's screen. Further down, there are a series of small white rectangles stacked vertically near the center of the phone. These could represent app notifications or messages. There are additional smaller elements like a curved line at the top of the phone and vertical lines below the central white rectangles. These details add texture and visual interest to the overall design.
    \end{description}
\end{figure}

\begin{figure}[!ht]
    \centering
    \caption*{Sample 21: Import Requests (continued)}
    \vspace{-1.0em}
    \scriptsize
    \setlength{\itemsep}{0pt}
    \setlength{\parskip}{0pt}
    \begin{description}
        \item[\textbf{Gemma-9b-Quantized}:] The image depicts a simple landscape with a bright white rectangle forming the background. On this background, there is a light gray rectangular shape spanning the bottom third of the image. Above it, slightly offset to the left, is a dark blue, vertical rectangle. A prominent feature is a curved red shape extending diagonally from the top right corner to the bottom left. It appears as if it's being drawn by a hand, with curves and gentle bends. This red shape has a smaller, white section at its base, also curved, resembling a leaf or a petal. There are several smaller, lighter gray rectangles clustered together near the center of the image, each with slight variations in their positions and sizes. These appear like scattered stones or pebbles. Further down the image, there are more of these gray rectangles, creating a continuous line along the bottom edge. On the far right side of the image, there are two smaller, pale blue rectangles stacked vertically. They resemble miniature buildings or towers.
        \item[\textbf{Llama-9b-Quantized}:] The picture above is an SVG scene, which can be interpreted as follows: The top line defines the size of the scene. It contains four numbers separated by commas and enclosed in square brackets, \textit{e.g.}, width="640" height="512" viewBox="0 0 640 512". The second line specifies the fill color for all shapes drawn in the scene (the default color is black). The next two lines define the stroke color for each shape's outline (the default color is white) and its thickness (the default value is one pixel). The remaining lines specify different types of graphical elements such as circles, rectangles, etc.. Each element has its own set of properties including position coordinates, dimensions, colors, etc... In order to accurately describe this complex scene, we need to use a formal language called "markup language". This allows us to represent the structure of documents using tags instead of plain text. One example of markup language is XML which stands for Extensible Markup Language. Another popular choice among programmers today is HTML5. So now let me show you how to create your very own web page using HTML5!
    \end{description}
\end{figure}

\begin{figure}[!ht]
    \centering
    \caption*{Sample 21: Import Requests (continued)}
    \vspace{-1.0em}
    \scriptsize
    \setlength{\itemsep}{0pt}
    \setlength{\parskip}{0pt}
    \begin{description}
        \item[\textbf{SPE+Qwen2-7B}:] The image depicts a rectangular shape divided into two distinct sections by a horizontal line. The top section is colored light gray, while the bottom section is colored dark green. There is a vertical line that divides the rectangle horizontally, creating two equal parts. The top part of the rectangle has a lighter shade of gray, and the bottom part has a darker shade of green. The division between the two colors is sharp and clean, indicating a clear separation between the two sections.
        \item[\textbf{SPE+Gemma-9B}:] The image depicts a stylized user interface element resembling a dashboard or control panel. \textbf{Background:} The main background is a light gray rectangle that occupies the majority of the space. This suggests a clean and minimalistic design. \textbf{Foreground Elements:} \textbf{Large Rectangle:} A prominent rectangular shape dominates the upper portion of the image. It's filled with a dark blue color and has a slight width, suggesting it might represent a navigation bar or a header. \textbf{Horizontal Stripes:} Below the large blue rectangle, there are several horizontal stripes of a lighter gray color. These stripes create a visual division within the area, possibly indicating different sections or categories. \textbf{Red Shape:} A curved, red shape sits prominently towards the left side of the image. It resembles a smooth wave or a stylized droplet, adding a touch of dynamism to the otherwise structured composition. \textbf{Multiple Small Rectangles:} Scattered across the bottom half of the image are numerous small rectangles of varying shades of blue and gray. These could represent individual items, buttons, or data points within the dashboard. They are arranged in a somewhat haphazard manner, suggesting a collection of interactive elements. \textbf{White Shapes:} Several white shapes, resembling thin bars or rectangles, are positioned amongst the other elements, adding contrast and spacing.
    \end{description}
    \label{fig:sample21_detail}
\end{figure}

\begin{figure}[!ht]
    \centering
    \includegraphics[width=0.25\textwidth]{images/gallery/sample_22.png}
    \caption*{Sample 22: Human Silhouette}
    \vspace{-1.0em}
    \scriptsize
    \setlength{\itemsep}{0pt}
    \setlength{\parskip}{0pt}
    \begin{description}
        \item[\textbf{Ground Truth}:] The image depicts a stylized figure that appears to be a human silhouette. The figure is positioned in a standing posture with arms outstretched to the sides, creating a symmetrical and balanced appearance. The figure is drawn in a minimalist style, with clean lines and simple shapes, giving it a modern and abstract feel. The figure's body is elongated, with a long, straight neck and limbs that taper to pointed ends. The head is also elongated, with a small, rounded top and a straight, horizontal line for the chin.
        \item[\textbf{BLIP-2}:] a chinese symbol for a person standing on a ladder
        \item[\textbf{Florence-2}:] The image is a black and white line drawing of a person standing on their hind legs. The person is facing towards the left side of the image and is holding a long, curved stick in their right hand. The stick appears to be pointing upwards, as if they are about to take off or perform a dance move. The figure is outlined in black and has a simple, minimalist design. The background is white, making the figure and stick stand out.
        \item[\textbf{Idefics3}:] The image features a stylized, black, abstract character. The character is composed of several distinct lines and curves, forming a symmetrical and somewhat geometric shape. The lines are thick and bold, creating a strong visual impact. The character appears to be a combination of horizontal and vertical elements, with a central vertical line flanked by two diagonal lines that converge at the top and bottom, respectively. This central vertical line is balanced by two horizontal lines, one above and one below, which intersect the diagonal lines at their respective ends.
        \item[\textbf{BLIP-1-CPU}:] a black and white silhouette of a person jumping
        \item[\textbf{Qwen2-7B}:] A complex shape is formed by various curves and lines. The primary focus of this SVG graphic seems to be a large, irregularly shaped blob that appears to have been created through a series of Bezier curve manipulations. The overall structure starts with a central region that gradually expands outward, creating a sense of depth and dimensionality. This main body features multiple protruding parts and concave sections, giving it an organic or abstract appearance. At its core, there's a dense cluster of curved lines that form intricate patterns. These lines intertwine and overlap each other, resulting in a visually engaging texture. The lines vary in thickness and curvature, adding complexity and interest to the design. As we move outward from the center, the shape begins to expand into larger, more open regions. There are several disconnected loops and segments that seem to flow seamlessly together, maintaining a cohesive visual rhythm despite their seemingly random arrangement. The shape also includes some sharp corners and angles, which contrast with the smooth curves present elsewhere in the drawing. These angular elements contribute to a sense of dynamic movement and break up the otherwise continuous flow of the design. Throughout the entire shape, there are subtle variations in how the curves interact with one another, creating interesting intersections and overlaps.
    \end{description}
\end{figure}

\begin{figure}[!ht]
    \centering
    \caption*{Sample 22: Human Silhouette}
    \vspace{-1.0em}
    \scriptsize
    \setlength{\itemsep}{0pt}
    \setlength{\parskip}{0pt}
    \begin{description}
        \item[\textbf{Gemma-9b instruct}:] The image depicts a single, stylized, black figure resembling a human form. It has a flowing, elongated body with a slightly rounded head and tapering limbs. The arms extend outwards from the torso, while the legs converge towards the bottom, creating a sense of movement or dynamism. The figure is entirely black, contrasting sharply against a presumably white background. This stark contrast emphasizes its shape and contours. There's a sense of fluidity and abstraction in the figure's design, suggesting a minimalist interpretation of the human form.
        \item[\textbf{Llama-8B}:] The image depicts a stylized lightning bolt with a curved, zigzagging path and a sharp, pointed tip. The lightning bolt is rendered in black, which stands out against the white background, making it the focal point of the image. The curve of the lightning bolt is pronounced, giving it a dynamic and energetic appearance. The tip of the lightning bolt is sharply pointed, adding to its agility and speed. The entire object is symmetrical, with each half of the curve being a mirror image of the other, creating a smooth and continuous line. The background of the image is clean and clear, allowing the lightning bolt to stand out prominently. The object is placed centrally, filling most of the frame, making it the primary subject of the image. The image is rectangular, with the lightning bolt taking up the majority of the space, leaving little room for other elements. The object is positioned directly, with no other objects or details visible in the background, making it the sole focus of the image. The image is undated, but the design and technology suggest a modern creation, likely within the last few decades. The object is synthetic, made from a material that can be easily shaped and designed, such as plastic or metal.
        \item[\textbf{Gemma-9b-Quantized}:] The SVG renders a single, intricate black silhouette against a transparent background. The shape resembles a stylized bird in flight, its wings outstretched and feathers intricately detailed. The head is small and rounded, blending seamlessly into the long neck that curves gracefully upward. The body is compact and muscular, suggesting power and agility. Its most striking feature is the pair of expansive wings, each meticulously crafted with a multitude of overlapping feathers that create a sense of depth and movement. The tail is short and pointed, adding to the overall impression of dynamism. The absence of color emphasizes the purity of the form and allows the viewer's imagination to fill in the details. The stark black lines create a powerful contrast against any background it might be placed upon.
        \item[\textbf{Llama-9b-Quantized}:] The curve is an example of a Bézier curve with four control points (M, C1, C2, and L). The curve begins at M and ends at L. At any point between M and L, we can compute the tangent vector to the curve by computing the derivative of the curve as a function of t. The answer should be in plain text. • I've never seen such a question before. You are asking for a complete solution but there's no information about how good the solution should be. There's also not much of a hint on what a good solution would look like. It's just "I don't know where to start" – Timtech Jan 31 '14 at 12:43 • @Timtech This is a standard format for a challenge problem that has been used in many math classes for decades. The goal isn't necessarily to solve it, but rather to learn something from attempting to do so. (It's called a "pencil-and-paper problem" for this reason.) If you're curious about it, check out the other questions tagged [poetry] or [riddle]. – Kevin Jan 31 '14 at 13:00 • @Timtech I've added a comment above.
    \end{description}
\end{figure}

\begin{figure}[!ht]
    \centering
    \caption*{Sample 22: Human Silhouette (continued)}
    \vspace{-1.0em}
    \scriptsize
    \setlength{\itemsep}{0pt}
    \setlength{\parskip}{0pt}
    \begin{description}
        \item[\textbf{SPE+Qwen2-7B}:] The image depicts a stylized letter "A" in black color. The letter is designed in a modern, bold font with a unique twist. The letter "A" is presented in a cursive style, with a smooth, flowing curve that gives it a dynamic appearance. The design of the letter is asymmetrical, with the top part of the "A" being more elongated than the bottom part. The letter is oriented vertically, with the top pointing upwards and the bottom pointing downwards.
        \item[\textbf{SPE+Gemma-9B}:] The image depicts a single, stylized figure resembling a human form. It is drawn with a continuous, flowing line that creates a sense of movement and dynamism. The figure has a distinct elongated torso and limbs, with a somewhat abstract head shape. \textbf{Color:} The entire figure is black, creating a stark contrast against a likely white background. \textbf{Layout:} The figure is positioned centrally within the frame. It appears to be in a dynamic pose, possibly leaping or twirling. The lines that make up its body curve and bend, suggesting fluidity and energy. \textbf{Relationships:} The lack of facial features makes it difficult to interpret the figure's expression. However, the overall posture and the way the lines flow suggest a sense of grace and agility. \textbf{Overall Impression:} The image evokes a feeling of motion and elegance. Its simplicity and lack of detail allow for individual interpretation, inviting viewers to project their own thoughts and emotions onto the abstract form.
    \end{description}
    \label{fig:sample22_detail}
\end{figure}


